% Domänenanalyse Energiegemeinschaften
\section{Domänenanalyse Energiegemeinschaften}
    \subsection{Überblick über Energiegemeinschaften}
        Energiegemeinschaften sind ein Teil des österreichischen Erneuerbaren-Ausbau-Gesetzes (EAG).
        Bürger können mit diesen ihre produzierte Energie an Konsumenten verteilen die sich ebenso in der Gemeinschaft befinden.
        Hierbei wird die Infrastruktur eines Netzbetreibers verwendet, um die Energie an die Konsumenten zu liefern.
        Die Mitglieder der Gemeinschaft können dadurch Energie untereinander Handeln.
        Die Abrechnung wird hierbei von der Gemeinschaft übernommen.

        Die Mitglieder einer Gesellschaft lassen sich in folgende Rollen unterteilen.
        \begin{itemize}[]
            \item \textbf{Erzeuger}: Produzieren Energie, z.\,B. durch Photovoltaik.
            \item \textbf{Verbraucher}: Nutzen die erzeugte Energie.
            \item \textbf{Prosumer}: Kombinieren Erzeugung und Verbrauch.
        \end{itemize}

    \subsection{Use Case Diagramm}
        \begin{figure}[H]
            \centering
            \includegraphics[width=0.8\textwidth]{images/UseCase-Diagram.png} % Platzhalter – Diagramm bitte extern erstellen
            \caption{Use Case Diagramm für Energiegemeinschaft}
        \end{figure}

        \begin{table}[]
            \centering
            \begin{tabular}{l|l}
                \hline
                \textbf{Akteur} & \textbf{Use Cases} \\
                \hline
                Energie-Producer & Energie einspeisen in BEG \\
                & Einspeisemenge überwachen \\
                & Vergütung erhalten \\
                \hline
                Energie-Consumer & Energie aus BEG beziehen \\
                & Verbrauchsdaten einsehen \\
                & Rechnung erhalten \\
                \hline
                Energie-Netzbetreiber & Netzstabilität überwachen \\
                & Einspeise- und Verbrauchsdaten abrufen \\
                \hline
                Systemadministrator & Nutzer verwalten \\
                & Datenintegrität sicherstellen \\
                & Schnittstellen betreuen \\
                \hline
                Staat / Behörde & Datenzugriff für Monitoring \\
                & Förderungen verwalten \\
                & Einhaltung gesetzlicher Vorgaben prüfen \\
                \hline
            \end{tabular}
            \caption{Rollen und Aufgaben in einer Bürgerenergiegemeinschaft}
            \label{tab:beg-akteure}
        \end{table}


    \subsection{Priorisierte Anforderungsliste}
        \subsubsection{Funktionale Anforderungen}

            \begin{table}[h]
                \centering
                \begin{tabular}{|l|p{4cm}|p{8cm}|}
                    \hline
                    \textbf{Priorität} & \textbf{Anforderung} & \textbf{Begründung} \\
                    \hline
                    Hoch & Datenerfassung von Smart Metern & Grundlage für alle weiteren Prozesse wie Abrechnung und Bilanzierung \\
                    \hline
                    Hoch & Abrechnungslogik innerhalb der Gemeinschaft & Zentrale Funktion zur finanziellen Transparenz und Fairness \\
                    \hline
                    Mittel & Verwaltung von Teilnehmern & Wichtig für Rollenmanagement, aber weniger zeitkritisch \\
                    \hline
                    Mittel & Energiebilanzierung & Relevant für Analyse und Optimierung, aber abhängig von Messdaten \\
                    \hline
                    Niedrig & Reporting und Visualisierung & Unterstützt Usability und Transparenz, aber nicht kritisch für MVP \\
                    \hline
                \end{tabular}
                \caption{Funktionale Anforderungen}
            \end{table}

        \subsubsection{Nicht-funktionale Anforderungen}

            \begin{table}[h]
                \centering
                \begin{tabular}{|l|p{4cm}|p{8cm}|}
                    \hline
                    \textbf{Priorität} & \textbf{Anforderung} & \textbf{Begründung} \\
                    \hline
                    Hoch & Datenschutz und Datensicherheit & Gesetzlich verpflichtend (DSGVO), besonders bei personenbezogenen Daten \\
                    \hline
                    Hoch & Verfügbarkeit und Ausfallsicherheit & System muss zuverlässig laufen, besonders bei Echtzeitdaten \\
                    \hline
                    Mittel & Skalierbarkeit & Wichtig für zukünftiges Wachstum, aber initial begrenzt relevant \\
                    \hline
                    Mittel & Performance (Latenz, Durchsatz) & Relevant für Nutzererlebnis, aber abhängig vom konkreten Use Case \\
                    \hline
                    Niedrig & Interoperabilität mit bestehenden Systemen & Nice-to-have, aber nicht zwingend für initiale Version \\
                    \hline
                \end{tabular}
                \caption{Nicht-funktionale Anforderungen}
            \end{table}
