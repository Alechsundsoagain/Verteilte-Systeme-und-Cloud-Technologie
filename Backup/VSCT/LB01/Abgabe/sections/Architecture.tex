\section{Softwarearchitektur}
    \subsection{Service-Identifikation}
        Basierend auf Domain-Driven Design Prinzipien wurden folgende Microservices identifiziert:

        \begin{itemize}[]
        \item \textbf{Community Management Service}: Verwaltung von Teilnehmern, Rollen und Beziehungen.
        \item \textbf{Billing Service}: Abrechnung, Fakturierung und Zahlungsstatus.
        \item \textbf{Metering Service}: Erfassung und Aggregation von Smart Meter-Daten.
        \item \textbf{Analysis Service}: Visualisierung, Analyse und Export von Energiedaten.
        \item \textbf{Core Service}: koordiniert Services
        \end{itemize}

        Jeder Service besitzt eine eigene Datenbankinstanz (MongoDB) und kommuniziert über RabbitMQ (asynchron) sowie REST (synchron).

    \subsection{Architekturdiagramme}
        \subsubsection*{System Context Diagram (C4 Level 1)}
            \begin{figure}[H]
                \centering
                \includegraphics[width=0.9\textwidth]{images/SystemContextDiagram.png} % Diagramm extern erstellen
                \caption{Systemkontextdiagramm der Energiegemeinschafts-Plattform}
            \end{figure}

        \subsubsection*{Sequenzdiagramm: Energieabrechnung}
            \begin{figure}[H]
                \centering
                \includegraphics[width=0.9\textwidth]{images/SequenzDiagram.png} % Diagramm extern erstellen
                \caption{Sequenzdiagramm für Energieabrechnung}
            \end{figure}

    \subsection{Kommunikationsdesign}
        \subsubsection*{Synchrone Kommunikation}
            \begin{itemize}[]
                \item REST APIs sowie für Benutzerinteraktion und Admin-Funktionen als auch für interne Service-Kommunikation
            \end{itemize}

        \subsubsection*{Asynchrone Kommunikation über RabbitMQ}
            \begin{itemize}[]
                \item \textbf{Exchange Types}: Topic Exchange für flexible Routing-Strategien
                \item \textbf{Queue Design}: Dedizierte Queues pro Service, z.\,B. \texttt{billing.queue}, \texttt{metering.queue}
                \item \textbf{Message Schemas}: JSON-basierte Payloads mit Validierung via JSON Schema
            \end{itemize}

    \subsection{Technologieentscheidungen}

        \begin{table}[H]
            \centering
            \begin{tabular}{@{}llp{7cm}@{}}
                \toprule
                \textbf{Komponente} & \textbf{Technologie} & \textbf{Begründung} \\ \midrule
                Programmiersprache & C\# mit .NET 9 & Moderne Sprache mit guter Microservice-Unterstützung und Performance \\
                Datenbank          & MongoDB        & Schema-flexibel, ideal für heterogene Energiedaten \\
                Message Broker     & RabbitMQ       & Etabliert, unterstützt Topic Exchange und hohe Zuverlässigkeit \\
                Containerisierung  & Docker         & Standard für portable Deployments \\
                Orchestrierung     & k3s mit k3d    & Lightweight Kubernetes für lokale Entwicklung \\
                API-Kommunikation  & REST           & REST für externe Clients und interne Services \\
                \bottomrule
            \end{tabular}
            \caption{Technologieentscheidungen und Begründungen}
        \end{table}
