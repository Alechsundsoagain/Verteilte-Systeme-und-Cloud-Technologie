\section{Setup-Dokumentation}

    \subsection{Entwicklungsumgebung}

        Für die lokale Entwicklungsumgebung wurden folgende Tools installiert und konfiguriert:

        \begin{itemize}[]
            \item Docker Desktop (Version 24.) mit WSL2-Integration (Ubuntu)
            \item .NET 9 SDK
            \item kubectl, helm, Git
            \item JetBrains Rider (IDE)
            \item MongoDB Compass (DataBase Viewer)
            \item Postman (API Network)
            \item Lens IDE (Cybernetes Cluster)
        \end{itemize}

    \subsection{k3s Cluster Setup mit k3d}
        Ein k3s Cluster wurde erfolgreich mit k3d erstellt:
        \begin{figure}[h]
            \centering
            \includegraphics[width=0.4\textwidth]{images/kubectl_getnodes_getnamespaces.png}
            \caption{Screenshot der laufenden k3s Nodes}
        \end{figure}
        \begin{figure}[h]
            \centering
            \includegraphics[width=0.4\textwidth]{images/services.png}
            \caption{Screenshot der laufenden Services}
        \end{figure}

    \subsection{Probleme und Lösungen}
        \begin{itemize}[]
            \item \textbf{RabbitMQ und MongoDB Installation fehlgeschlagen} Docker Login auch in der WSL nötig gewesen.
            \item \textbf{Port Forwarding hat nicht funktioniert} In 2h nochmal probiert.
            \item \textbf{Kubernetes wurden von Lens nicht erkannt:} Herauskopieren der konfiguration von innerhalb von WSL.
            \item \textbf{MongoDB nicht verbindbar von außerhalb des Containers:} Verwendung von root und secretpass.
            \item \textbf{Veraltete WSL Version + Fehler in der Deinstallation:} Wiederholtes updaten.
            \item \textbf{Cluster-Commands darf man nicht aus adobe kopieren:} Datei in SumatraPDF öffnen und von dort kopieren.
            \item \textbf{:} 
        \end{itemize}
